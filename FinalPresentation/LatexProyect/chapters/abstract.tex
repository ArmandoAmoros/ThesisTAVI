% !TeX root = /../Report.tex

\chapter*{Abstract\markboth{Abstract}{Abstract}}
\addcontentsline{toc}{chapter}{\protect\numberline{}Abstract}
\label{abstract}

This thesis work intends to find the most suitable master device type for a TAVI/TAVR teleoperated robot, capable of controlling the 2 DOF (translation and rotation) of the different catheters and guide wires used during the intervention.\\

After a state-of-the-art research in catheter handling teleoperated robots, four different master devices were tested. Each device controls each one of the 2 DOF independently: The first device with completely digital inputs (Keyboard type), the second device hybrid with 1 digital input (translation) and 1 analogical input (rotation) (Remote Controller type), and the remaining two devices with completely analogical inputs (Joystick type and CatheterLike type).\\

The experiments were performed by 15 candidates from which 1 was an expert TAVI surgeon. Each device was tested under three different experiments and by the appreciation of the users, the first two experiments were designed as a follow the target task assessing the precision and response of each DOF independently. The third experiment was designed as a navigation task, using both DOF, measuring the time and smoothness of the movements and path followed until reaching the goal.\\

The results of the experiments and the user's poll responses indicate that the Joystick type device has a better overall performance for controlling the 2 DOF of a regular catheter used in TAVI/TAVR surgery.\\
